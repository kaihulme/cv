\cvsection{Projects}
\vspace{-0.1cm}

\begin{cvprojects}

% %---------------------------------------------------------
% \cvproject
%     {\href{https://github.com/kaihulme/ad-gan}{\faGithubSquare\acvHeaderIconSep ad-gan}} % Project location
%     {GANs for Alzheimer's Disease Detection} % Project Name
%     {\begin{cvitems}
%         \item Thesis project studying the use of GANs as an augmentation technique for improved Alzheimer’s Disease detection in MRI volumes.
%         \item Using the OASIS dataset of Alzheimer's Disease patients, built neuroimaging pre-processing pipelines with NiPype to use data with ML models.
%         \item Built a method for Alzheimer's Disease classification in MRI volumes using a slice extraction method and CNN model using TensorFlow / Keras.
%         \item Implementation of Generative Adversarial Networks in TensorFlow / Keras with custom training logic and loss functions.
%         \item Implementation of current literature on GAN models and machine learning Alzheimer's Disease detection methods.
%         \item Model deployment to Nvidia GPU using Docker containerisation.
%       \end{cvitems}
%     }

%---------------------------------------------------------
\cvproject
    {\href{https://github.com/kaihulme/ad-gan}{\faGithubSquare\acvHeaderIconSep ad-gan}} % Project location
    {GANs for Alzheimer's Disease Detection} % Project Name
    {\begin{cvitems}
        \item Thesis project studying the use of GANs as an augmentation technique for improved Alzheimer’s Disease detection in MRI volumes.
        \item Developed neuroimaging pre-processing pipelines with NiPype to extract brain regions from MRI volumes for use with ML models.
        \item CNN classification of sliced MRI volumes and implementation of GANs in TensorFlow / Keras through custom training logic.
        \item Built as Python package with containerised CUDA environment for GPU deployment using Nvidia-Docker.
      \end{cvitems}
    }
    
%---------------------------------------------------------
\cvproject
    {\href{https://github.com/kaihulme/nestpi}{\faGithubSquare\acvHeaderIconSep nestpi}} % Project location
    {NestPi} % Project Name
    {\begin{cvitems}
        \item RasberryPi nest box camera and Flask web app for monitoring, hosted by AWS EC2 instance with RDS SQL database and S3 for video storage.
        \item CI/CD pipeline implemented with GitHub actions automates testing with Pytest and build deployment to AWS services using Elastic Beanstalk.
        \item Containerised Gunicorn application, with automated publishing to DockerHub and AWS instance utilising docker-compose.      \end{cvitems}
    }
    
%---------------------------------------------------------
% \cvproject
%     {\href{https://github.com/kaihulme/alspac-mhst}{\faGithubSquare\acvHeaderIconSep alspac-mhst}} % Project location
%     {Mental Health and Screen Time} % Project Name
%     {\begin{cvitems}
%         \item Data science team project for Jean Golding Institute, analysing ALSPAC data for correlation of screen usage and child mental health issues.
%         \item Mostly missing data led to analysis of various developed multivariate model-based imputation methods using Scikit-Learn and XGBoost.
%         \item Using classification methods and statistical analysis it was concluded, given the data, there was little to no evidence for such a correlation.
%         \item IEEE paper with full analysis and methods available on GitHub repository.
%       \end{cvitems}
%     }

\cvproject
    {\href{https://github.com/kaihulme/alspac-mhst}{\faGithubSquare\acvHeaderIconSep alspac-mhst}} % Project location
    {Mental Health and Screen Time} % Project Name
    {\begin{cvitems}
        \item Data science team project for Jean Golding Institute, analysing ALSPAC data for correlation of screen usage and child mental health issues.
        \item Developed various multivariate model-based imputation methods using Scikit-Learn and XGBoost to handle missing data.
        \item IEEE paper with full analysis and methods available on GitHub repository.
      \end{cvitems}
    }
    
%---------------------------------------------------------
% \cvproject
%     {\href{https://github.com/kaihulme/ml-analysis}{\faGithubSquare\acvHeaderIconSep ml-analysis}} % Project location
%     {Machine Learning Model Analysis} % Project Name
%     {\begin{cvitems}
%         \item Analysis and optimisation of various classification and regression models using Scikit-Learn, Keras and PyMC3.
%         \item Built visualisation and performance methods to compare support vector machines, neural networks and ensemble classification models.
%         \item Implemented Bayesian linear regression with PyMC3 to embed prior domain knowledge into the model learnt during exploratory data analysis.
%       \end{cvitems}
%     }
    
%---------------------------------------------------------
\cvproject
    {\href{https://github.com/kaihulme/dartboard-detection}{\faGithubSquare\acvHeaderIconSep dartboard-detection}} % Project location
    {Dartboard Detection} % Project Name
    {\begin{cvitems}
        \item {Dartboard detection using OpenCV and traditional computer vision techniques, implemented in C++ and Python.}
        \item {Bounding box detection from images using an ensemble of Viola-Jones cascades, Hough transform shape detectors and KMeans clustering.}
      \end{cvitems}
    }
    
% %---------------------------------------------------------
\cvproject
    {{\href{https://github.com/kaihulme/game-of-life}{\faGithubSquare\acvHeaderIconSep game-of-life}}} % Project location
    {Game of Life} % Project Name
    {\begin{cvitems}
        \item {Concurrent implementation of Game of Life in C (XC), using parallel programming concepts such as farming and inter-process communication.}
        \item{Embedded application for XMOS xCORE-200 with bit-manipulation optimisations to maximise limited resources.}
      \end{cvitems}
    }
    
% %---------------------------------------------------------
% \cvproject
%     {\href{https://github.com/kaihulme/os-kernel}{\faGithubSquare\acvHeaderIconSep os-kernel}} % Project location
%     {Operating System Kernel} % Project Name
%     {\begin{cvitems}
%         \item {Implementation of a UNIX-style operating system in C for the ARM architecture.}
%         \item {Multiprocessing OS with inter-process communication and priority-based scheduling.}
%       \end{cvitems}
%     }

%---------------------------------------------------------

\end{cvprojects}
